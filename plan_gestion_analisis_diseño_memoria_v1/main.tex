% Preamble del documento

\documentclass[11pt, a4paper, titlepage]{article}


%%%%%%%%%%%%%%%%%%%%%%%%%%%%%%%%%%%%%%%%%%%%%%%%%%%%%%%%%%%%%%%%%%%%%%%%%%%%%%%%%%%%%%%%%%%%%%%%
% Encoding e idioma Español, 
%%%%%%%%%%%%%%%%%%%%%%%%%%%%%%%%%%%%%%%%%%%%%%%%%%%%%%%%%%%%%%%%%%%%%%%%%%%%%%%%%%%%%%%%%%%%%%%%
\usepackage[utf8]{inputenc}
\usepackage[spanish, es-tabla]{babel} % Para poner cuadro en vez de tabla, etc
%%%%%%%%%%%%%%%%%%%%%%%%%%%%%%%%%%%%%%%%%%%%%%%%%%%%%%%%%%%%%%%%%%%%%%%%%%%%%%%%%%%%%%%%%%%%%%%%



%%%%%%%%%%%%%%%%%%%%%%%%%%%%%%%%%%%%%%%%%%%%%%%%%%%%%%%%%%%%%%%%%%%%%%%%%%%%%%%%%%%%%%%%%%%%%%%%
% Paquetes 
%%%%%%%%%%%%%%%%%%%%%%%%%%%%%%%%%%%%%%%%%%%%%%%%%%%%%%%%%%%%%%%%%%%%%%%%%%%%%%%%%%%%%%%%%%%%%%%%
\usepackage[colorlinks=true, urlcolor=cyan, citecolor=blue, linkcolor=blue, hidelinks]{hyperref} % /url{}, /ref{}...
\usepackage[a4paper,top=2cm,bottom=2cm,left=3cm,right=3cm,marginparwidth=1.75cm]{geometry}
\usepackage{minted} % Código
\usepackage{setspace}

\usepackage[sorting=none, style=numeric]{biblatex} % aparece 1ro la 1ra citada
\addbibresource{citas.bib}

\usepackage{amsmath}

\usepackage[utf8]{inputenc} 

\usepackage[table]{xcolor} % Colores en tablas/texto/etc

\usepackage{graphicx} % Imágenes

\usepackage{tabularx} % Tablas más potentes
\usepackage{longtable} % Tablas muy largas
\usepackage{multirow}
\usepackage{booktabs}
\usepackage{multicol}
\usepackage{adjustbox} % Ajustar tablas al ancho de la página

\usepackage{datetime} % Fechas
\usepackage{svg} % para el svg del logo de la eina
\usepackage{titlesec}
\usepackage[section]{placeins}  % para floatbarrier en secciones, 
                                % con [section] se ponen silenciosamente en cada seccion

\usepackage{listliketab}                                
\usepackage{fancyhdr} %pagestyles adicionales

\usepackage{lmodern} % Fuente moderna, \code con negrita
\usepackage{caption} % para poner labels en tabular sin convertirlos en tablas que flotan

\makeatletter
%%%%%%%%%%%%%%%%%%%%%%%%%%%%%%%%%%%%%%%%%%%%%%%%%%%%%%%%%%%%%%%%%%%%%%%%%%%%%%%%%%%%%%%%%%%%%%%%
\renewcommand\paragraph{\@startsection{paragraph}{4}{\z@}%
            {-2.5ex\@plus -1ex \@minus -.25ex}%
            {1.25ex \@plus .25ex}%
            {\normalfont\normalsize\bfseries}} % no sé de donde ha salido esto
\makeatother
\setcounter{secnumdepth}{4} % how many sectioning levels to assign numbers to
\setcounter{tocdepth}{4}    % how many sectioning levels to show in ToC
%\setlength{\parindent}{0pt} % para quitar indentación inicial en párrafos

%%%%%%%%%%%%%%%%%%%%%%%%%%%%%%%%%%%%%%%%%%%%%%%%%%%%%%%%%%%%%%%%%%%%%%%%%%%%%%%%%%%%%%%%%%%%%%%%
% Título para \maketitle
%%%%%%%%%%%%%%%%%%%%%%%%%%%%%%%%%%%%%%%%%%%%%%%%%%%%%%%%%%%%%%%%%%%%%%%%%%%%%%%%%%%%%%%%%%%%%%%%
\title{Plantilla de trabajo}
\author{autor1 \and autor2 \and autor3}

% Macro para mostrar el mes y el año (https://texnique.fr/osqa/questions/1359/commandes-month-year)
\def\monthyear{\ifcase\month\or
  Enero\or Febrero\or Marzo\or Abril\or Mayo\or Junio\or
  Julio\or Agosto\or Septiembre\or Octubre\or Noviembre\or Diciembre\fi
  \space\number\year}

\date{\monthyear}
%%%%%%%%%%%%%%%%%%%%%%%%%%%%%%%%%%%%%%%%%%%%%%%%%%%%%%%%%%%%%%%%%%%%%%%%%%%%%%%%%%%%%%%%%%%%%%%%


%%%%%%%%%%%%%%%%%%%%%%%%%%%%%%%%%%%%%%%%%%%%%%%%%%%%%%%%%%%%%%%%%%%%%%%%%%%%%%%%%%%%%%%%%%%%%%%%
% Estilos
%%%%%%%%%%%%%%%%%%%%%%%%%%%%%%%%%%%%%%%%%%%%%%%%%%%%%%%%%%%%%%%%%%%%%%%%%%%%%%%%%%%%%%%%%%%%%%%%
%\pagestyle{fancy} % con el nombre de la sección y página arriba
%\fancyhf{}
%\pagenumbering{arabic}
%\rfoot{Page \thepage}
\newcommand{\code}{\texttt} % para poner codigo en Monospace


\pagestyle{fancy}
\fancyhf{}
%\rhead{\rightmark}
\lhead{\leftmark}
\rfoot{\thepage}
%%%%%%%%%%%%%%%%%%%%%%%%%%%%%%%%%%%%%%%%%%%%%%%%%%%%%%%%%%%%%%%%%%%%%%%%%%%%%%%%%%%%%%%%%%%%%%%%

% Fin del preamble

% Forza las opciones desde el principio del documento si no se aplican
\AtBeginDocument{
  \urlstyle{same}
  \addtocontents{toc}{\small}
  \addtocontents{lof}{\small}
}
% Cuerpo del documento
\begin{document} 


%%%%%%%%%%%%%%%%%%%%%%%%%%%%%%%%%%%%%%%%%%%%%%%%%%%%%%%%%%%%%%%%%%%%%%%%%%%%%%%%%%%%%%%%%%%%%%%%%		
% PORTADA
%%%%%%%%%%%%%%%%%%%%%%%%%%%%%%%%%%%%%%%%%%%%%%%%%%%%%%%%%%%%%%%%%%%%%%%%%%%%%%%%%%%%%%%%%%%%%%%%
\begin{titlepage}
\thispagestyle{empty}
\vspace*{1.7mm}

\fontsize{28pt}{28pt}\selectfont
\begin{center}
Plan de gestión, análisis, diseño y memoria del proyecto
\end{center}
\fontsize{24pt}{24pt}\selectfont
\vspace*{4mm}
\begin{center}
Grupo 01 - Grace Hopper
\end{center}
\vspace*{17.7mm}
\baselineskip 36pt
\begin{center}
\fontsize{12pt}{12pt}\selectfont
\center{\rm  Autores}

\vspace*{3.65mm} 
\fontsize{18pt}{18pt}\selectfont
\center{García Vázquez, Samuel,  720873\\
        Crespo Aznárez, Francisco José, 798066\\
        Estevan Tomás, Emilio, 798516\\
        Grima Terrén, Jorge 801324\\
        Enguita Lahoz, Guillermo, 801618\\
        González Pizarro, Laura,  805467\\
        }
\vspace*{25mm} 
\href{https://github.com/UNIZAR-30226-2022-01}{\color{blue}{Organización de Github}}
%\baselineskip 36pt
%\fontsize{12pt}{12pt}\selectfont
%\center{\rm  Profesores}
%\vspace*{3.65mm}
%\fontsize{14pt}{14pt}\selectfont
%\center{Nombre profesor 1}
\end{center}

\setcounter{footnote}{0}
%\vspace*{3.65mm}
%\fontsize{12pt}{12pt}\selectfont
%\begin{center}
%\monthyear
%\end{center}
\vspace*{10.7mm}
\begin{figure}[!h]
  \centering
	\includesvg[width=69.62mm]{eina}
\end{figure}

\renewcommand{\thefootnote}{\arabic{footnote}}
%\pagenumbering{gobble}

\end{titlepage}
\newpage
%%%%%%%%%%%%%%%%%%%%%%%%%%%%%%%%%%%%%%%%%%%%%%%%%%%%%%%%%%%%%%%%%%%%%%%%%%%%%%%%%%%%%%%%%%%%%%%%





%%%%%%%%%%%%%%%%%%%%%%%%%%%%%%%%%%%%%%%%%%%%%%%%%%%%%%%%%%%%%%%%%%%%%%%%%%%%%%%%%%%%%%%%%%%%%%%%%		
% CUERPO
%%%%%%%%%%%%%%%%%%%%%%%%%%%%%%%%%%%%%%%%%%%%%%%%%%%%%%%%%%%%%%%%%%%%%%%%%%%%%%%%%%%%%%%%%%%%%%%%
\thispagestyle{empty}
\fontsize{11pt}{11pt}\selectfont

\setcounter{tocdepth}{2}

% Índice con links en negro y no en azul
{
    \hypersetup{linkcolor=black}
    \doublespacing
    \tableofcontents
}

\thispagestyle{empty}

% Texto
\clearpage
\setcounter{page}{1}
\section{Introducción}
A lo largo del documento vamos a presentar un proyecto que hemos denominado World Domination. Se trata de un juego de estrategia on-line basado en turnos e inspirado en Risk, en el que cada jugador lucha con el resto por la dominación del mundo. Cada uno de los jugadores cuenta con un número de soldados y territorios controlados por los mismos, así como con una baraja de cartas propia que serán canjeadas por más soldados. En cada turno, los jugadores tendrán que batallar con soldados bajo el dominio de otros jugadores, decidiendose los resultados mediante tiradas de dados. \newline

El juego proporcionará funcionalidades adicionales a los jugadores, que les permitirá disfrutar de una experiencia social en el entorno del juego, fomentando la competición entre ellos y una mayor participación en futuras partidas. \newline

El sistema debe ser capaz de gestionar las cuentas de los usuarios junto con las partidas que se desarrollen, así como la gestión de turnos y la inicialización y finalización de cada una de las partidas. Algunos de los objetivos fundamentales del sistema son: 

\begin{itemize}
\item Se permitirá a los usuarios registrarse e iniciar sesión con sus credenciales, que constan de un email y contraseña. El uso de una cuenta de usuario será necesario para jugar partidas.

\item El sistema contará con un ranking global de los mejores jugadores y de sus respectivas puntuaciones de las partidas jugadas.

\item Se permitirá a los usuarios personalizar elementos del juego en función de sus preferencias, como las fichas, su foto de perfil, o los dados usados durante las partidas.

\item El sistema recompensará al usuario por haber jugado o ganado un número determinado de partidas con puntos canjeables en la tienda del juego. La tienda será el punto de compra de personalizaciones de los elementos del juego.

\item Se permitirá gestioar una lista de amigos de cada usuario.

\item El sistema permitirá crear partidas públicas y privadas, conforme a las siguientes reglas:
\begin{itemize}
    \item Las Partidas serán de 2 a 6 jugadores.
    \item Se permitirá a cualquier usuario entrar a partidas públicas.
    \item En el caso de partidas privadas, solo se permitirá la entrada con una contraseña, introducida al crearla.
    \item Se encontrará disponible un chat en la partida.
    \item Se restringirá al usuario a poder jugar solo una partida al mismo tiempo.
\end{itemize}

\item Las partidas se desarrollarán de forma asíncrona, permitiendo a los usuarios cuyo turno les corresponda responder en cualquier momento. Si el usuario se encuentra conectado, podrá ver una notificación en la pantalla. En caso de que el usuario no se encuentre conectado, se enviará un correo al mismo notificando que puede realizar la jugada.

\item Habrá un mecanismo de expulsión de jugadores que no respondan o realicen jugadas en un tiempo determinado.

\end{itemize}

A lo largo de la realización del proyecto, será necesario realizar algunas entregas del trabajo llevado a cabo, en reuniones propuestas en unos plazos establecidos:\newline
Una reunión para la evaluación de las funcionalidades clave ¿…?\newline
Una reunión para la evaluación de todas las funcionalidades del juego, con el sistema implementado en una versión final, con el objetivo de realizar ajustes finales, en la semana del 16 al 20 de Mayo.\newline
El sistema finál se entregará no más tarde del 1 de junio, con los cambios propuestos durante las reuniones intermedias.

A continuación hablaremos sobre otros aspectos del proyecto, de la gestión, la organización, análisis de requisitos, así como del diseño del sistema.

\section{Organización del proyecto}
El equipo está formado por 6 integrantes: Samuel García, Guillermo Enguita, Laura Gonzalez, Francisco Crespo, Emilio Estevan y Jorge Grima.
Hemos dividido el proyecto en tres bloques fundamentales:\newline
\begin{itemize}
    \item Backend mediante \textbf{GoLang}.
    \item Primera versión frontend mediante \textbf{React}.
    \item Segunda versión frontend mediante \textbf{Angular}.\newline
\end{itemize}

El trabajo del primer bloque será llevado a cabo por Samuel y Guillermo, el del segundo bloque por Francisco y Emilio, mientrás que el del tercer bloque será realizado por Laura y Jorge. 
Aunque cada uno tenga un bloque asignado, en todo momento se ayudará a integrantes encargados del trabajo en otros bloques si fuera necesario. \newline

Además, el equipo decidió nombrar como director a Francisco Crespo, que será quién se encargue de llevar a cabo la organización principal del proyecto, así como de las entregas necesarias del trabajo.

\section{Plan de gestión del proyecto}

A continuación, se describe cómo se llevaran a cabo distintas tareas a realizar en diferentes momentos del proyecto, como pueden ser: plan inicial de trabajo, división del trabajo a realizar o control de creación de software, entre otros.

\subsection{Inicio del proyecto}

\begin{itemize}

\item En primer lugar, en referencia a formación común del equipo y, debido a que la completitud de la documentación necesaria para el proyecto va a ser realizada en \textbf{Latex}, todos los miembros del equipo han adquirido conocimientos básicos para la utilización y explotación de esta herramienta. Para lograr este objetivo, se han realizado dos tutoriales. El primero de ellos <<Learn LaTeX in 30 minutes>>\textsuperscript{\cite{latextutorial1} }se puede encontrar en la página web de \textit{Overleaf}. A su vez, se han seguido los manuales de ayuda proporcionados por la página oficial de \textit{Latex}\textsuperscript{\cite{latextutorial2}}. 

\item Por otro lado, continuando en lo que respecta a la formación común, los integrantes del Grupo 1 asistieron a la práctica: \textit{Puesta en marcha de GitHub y fundamentos de Git} impartida por los profesores de la asignatura \textit{Proyecto Software}, la cual consistió en familiarizarse con la herramienta \textbf{Git} y su uso en conjunto con \textbf{GitHub}. A su vez, por medio del tutorial: \textit{Git, GitHub y Publicación Web}\textsuperscript{\cite{githubtutorial}} se habían adquirido los conocimientos básicos de la respectiva tecnología.

\item Los encargados del desarrollo del \textit{backend} harán uso de \textbf{GoLang} como lenguaje de programación. Se trata de un lenguaje bastante conocido debido a que fue utilizado durante las prácticas de la asignatura \textit{Sistemas Distribuidos}. De cara a fijar conocimientos y a aprender nuevas mecánicas, fue realizado el tutorial de la página web oficial de GoLang. \\
Por otro lado, también realizaron una formación relacionada con las distintas \textbf{APIs web}, concretamente el tutorial \textit{Introducción a las APIs web} de MozillaDeveloper\textsuperscript{\cite{apisweb}}.

\item El equipo encargado del primer \textit{frontend}, aplicación web con \textbf{React}, necesita adquirir los conocimientos necesarios para trabajar con esta tecnología. Para ello, se realizará el tutorial oficial de \textbf{ReactJs}: \textit{Tutorial: Introducción a React}\textsuperscript{\cite{reactjs}}, el cual consta de una duración estimada de 8 horas. A su vez, será realizado el tutorial de APIs web anteriormente mencionado\textsuperscript{\cite{apisweb}}.

\item Por último, el equipo encargado de desarrollar el segundo \textit{frontend}, aplicación web con \textbf{Angular}. Debido a que no se ha trabajado con este framework es necesario realizar un tutorial para aprender a usar esta tecnología. De cara a lograr este objetivo, será realizado el tutorial \textit{Angular y TypeScript}\textsuperscript{\cite{angular}} proporcionado por DesarrolloWeb, el cual tiene un tiempo estimado de realización próximo a las 7 horas.

\end{itemize}

\subsection{Control de configuraciones, construcción del software y aseguramiento de la calidad del producto}

Se han establecido los siguientes procedimientos relativos a las tareas de construcción del software, gestión de incidencias, automatización,  calidad y despliegue del sistema por completo.

\subsubsection{Convenciones y procedimientos para la construcción y control del software}

\begin{itemize}
    \item La organización y nombrado de los ficheros fuente seguirá las prácticas recomendadas por cada uno de los frameworks a utilizar. De este modo, se seguirán las siguientes guías de estilo, nombrado de ficheros y organización de proyectos:
    \begin{itemize}
        \item Para Angular, se utilizará la guía de estilo oficial\textsuperscript{\cite{estiloangular}} y la guía de estructura de proyectos propuesta por la documentación oficial\textsuperscript{\cite{estructuraangular}}.
        
        \item Para React, se utilizará la
        guía de estilo de airbnb\textsuperscript{\cite{estiloreact}} y la estructura de proyectos propuesta por la documentación oficial\textsuperscript{\cite{estructurareact}}.
        
        \item Para golang, se utilizará el estilo impuesto por la herramienta automática de formato de dicho lenguaje\textsuperscript{\cite{estilogolang}} y la estructura de proyectos recomendada por la comunidad\textsuperscript{\cite{estructuragolang}}, compatible con las herramientas de compilación del lenguaje.
    \end{itemize}
    
    \item Los ficheros de documentación no interna serán nombrados con minúsculas y podrán contener guiones bajos, sin caracteres especiales. Si forman parte de la documentación del código o interfaces, su nombre empezará por \textbf{documentacion\_}, y si forman parte de guías de uso para los clientes, su nombre empezará por \textbf{guia\_}.
    
    \item Todo el código será subido a los repositorios de la organización de Github\footnote{\href{https://github.com/UNIZAR-30226-2022-01}{\color{blue}{Repositorio de la organización de Github}}}, donde todos los miembros tendrán permisos de acceso, modificación y administración de los mismos. Los repositorios serán los siguientes:
    \begin{itemize}
        \item Un repositorio para el backend del sistema\footnote{\href{https://github.com/UNIZAR-30226-2022-01/proyecto_software_backend}{\color{blue}{Repositorio del backend del sistema}}}, en el cual se localizará todo el código y ficheros para el despliegue relativo al servidor de API escrito en golang.
        
        \item Un repositorio para el frontend de React\footnote{\href{https://github.com/UNIZAR-30226-2022-01/proyecto_software_frontend_react}{\color{blue}{Repositorio del frontend de React}}}, en el cual se localizará todo el código y ficheros para el despliegue relativo al frontend de React.
        
        \item Un repositorio para el frontend de Angular\footnote{\href{https://github.com/UNIZAR-30226-2022-01/proyecto_software_frontend_angular}{\color{blue}{Repositorio del frontend de Angular}}}, en el cual se localizará todo el código y ficheros para el despliegue relativo al frontend de Angular.
        
        \item Un repositorio para la documentación pública\footnote{\href{https://github.com/UNIZAR-30226-2022-01/proyecto_software_documentacion}{\color{blue}{Repositorio de documentación pública}}}, en el cual se localizarán todos los archivos de documentación interna o para los clientes finales y parciales, como archivos de \LaTeX.
    \end{itemize}
    
    \item La gestión de incidencias y tareas pendientes se realizará mediante el gestor de incidencias de Github del repositorio relacionado con ellas.
    
    \item Todas las tareas, funcionalidades a implementar o ampliar y errores a arreglar serán documentados en incidencias individuales con una descripción concisa, que formarán parte de hitos para entregas y versiones intermedias si aplica, y tendrán etiquetas asignadas acordemente.
    
    Las incidencias serán asignadas a quienes trabajen en ella para mantener un control de actividades y tiempo, y podrán ser cerradas y abiertas de nuevo libremente.
    
    \item Todos los commits a realizar no requerirán aprobación, pero deberán ser probados para que no causen problemas de compilación, y deberían ser tan funcionalmente completos como sea posible. 
    
    Además, no deberán abarcar un número excesivo de cambios en el código para favorecer que se puedan deshacer si es necesario, y deberán tener una descripción de los cambios realizados concisa.
    
    
\end{itemize}

\subsubsection{Convenciones y procedimientos para el despliegue del software}

\begin{itemize}
    \item Las herramientas necesarias para construir el software serán aquellas necesarias para desplegar los frontend de React y Angular y el backend de golang por separado, y Docker. \newline
    
    En el repositorio del backend se proporcionarán scripts de automatización del despliegue del backend con cada uno de los frontends sobre contenedores de Docker. También será posible elegir si se desea construir el contenedor para ejecutar y obtener resultados de pruebas unitarias o ejecutar el sistema en sí. \newline
    
    Dicho script de automatización se hará cargo de la compilación, obtención de dependencias y puesta en marcha y configuración de cualquier elemento adicional como la base de datos, con el objetivo de poder realizar pruebas localmente con una base común para todos los miembros del equipo.
    
    \item El despliegue en entornos de producción se realizará con contenedores de Docker en servidores cloud, con configuraciones como claves de acceso a la base de datos o rutas y puertos a utilizar a introducir manualmente como parte del despliegue del contenedor por motivos de seguridad.
\end{itemize}

\subsubsection{Convenciones y procedimientos para el control de la calidad del software}

\begin{itemize}
    \item Se seguirán buenas prácticas para la documentación del código, como las publicadas por Google\textsuperscript{\cite{documentaciongoogle}}, para la documentación de la API pública y de funciones auxiliares críticas.
    
    \item El diseño de la interfaz deberá ser lo más similar posible entre ambos frontends para mantener la consistencia y por tanto acordado entre ambas partes encargadas de ellos, y se seguirán los principios básicos de diseño de interfaces y usabilidad, como las heurísticas de Nielsen\textsuperscript{\cite{heuristicasnielsen}}.
    
    \item El código de cada una de las partes del proyecto será revisado por ambos miembros encargados de él. Adicionalmente, se prepararán tests automáticos para aquellos aspectos que pudieran requerir demasiado tiempo para probarlos.
    
    \item Las entregas a los clientes serán probadas íntegramente por todos los miembros del equipo en conjunto, y se garantizará que no existen problemas en las funcionalidades implementadas hasta dicho momento.
\end{itemize}

El responsable del seguimiento de todos estos aspectos será Samuel García.



\subsection{Calendario del proyecto, división del trabajo, coordinación, comunicaciones, monitorización y seguimiento}

\section{Análisis y diseño del sistema}

\subsection{Análisis de requisitos}
\\ Los requisitos funcionales de nuestra aplicación se pueden encontrar detallados en la tabla \ref{tab:rf}, así mismo, la tabla \ref{tab:rnf} recoge los requisitos no funcionales. Ambos tipos de requisitos se encuentran claramente detallados e identificados, para facilitar su referencia en documentación futura.
% La leyenda no cabe bien si está debajo de la tabla y la pone donde quiere
\begin{longtable}
    \centering
    \begin{tabularx}{\textwidth}{|l|X|}
    \hline
         Código & Descripción  \\
         \hline
         RF-1 & El sistema permitirá a los usuarios registrarse, utilizando un correo electrónico y estableciendo un nombre de usuario y una contraseña.\\
         RF-2 & El sistema permitirá a los usuarios iniciar sesión, utilizando su correo electrónico y su contraseña.\\
         RF-3 & El sistema permitirá a los usuarios iniciar sesión, utilizando su nombre de usuario y su contraseña.\\
         RF-4 & El sistema proporcionará a los usuarios un juego de Risk.\\
         RF-5 & El sistema ofrecerá a los usuarios un lobby que permitirá buscar partidas públicas en curso, destacando las partidas en las que participen amigos del usuario.\\
         RF-6 & El sistema recompensará a los jugadores al ganar partidas, otorgándoles puntos canjeables.\\
         RF-7 & El sistema ofrecerá una tienda a los usuarios en la que podrán utilizar los puntos descritos en el requisito RF-6 para comprar diferentes cosméticos que permitan la personalización. \\
         RF-8 & El sistema permitirá a los usuarios personalizar elementos del juego.\\
         RF-8.1 & El sistema permitirá a los usuarios modificar su foto de perfil y su biografía.\\
         RF-8.2 & El sistema permitirá a los usuarios personalizar sus dados y fichas de juego, utilizando los cosméticos comprados en la tienda definida en el requisito RF-7.\\
         RF-8.3 & Las personalizaciones realizadas serán visibles por todos los usuarios.\\
         RF-9 & El sistema permitirá a los usuarios gestionar una lista de amigos.\\
         RF-10 & El sistema permitirá a los usuarios consultar una clasificación global, la cuál mostrará a los mejores jugadores junto con sus respectivas puntuaciones.\\
         RF-11 & El sistema permitirá a los usuarios consultar el perfil personal de otros jugadores.\\
         RF-12 & El sistema permitirá a los usuarios crear una partida.\\
         RF-12.1 & El sistema permitirá crear partidas públicas, a las que cualquier usuario se podrá unir.\\
         RF-12.2 & El sistema permitirá crear partidas privadas, de forma que el usuario deberá definir una contraseña para el acceso a dicha partida.\\
         RF-12.3 & El sistema permitirá al usuario establecer el número de jugadores de la partida (de 2 a 6 jugadores).\\
         RF-13 & Las partidas se desarrollarán de forma asíncrona.\\
         RF-13.1 & Cuando llegue el turno del jugador, el sistema le avisará mostrando una notificación en la pantalla si está conectado a la aplicación.\\
         RF-13.2 & Si el usuario no está conectado, el sistema enviará el aviso a través de correo electrónico.\\
         RF-13.3 & Si el usuario no realiza su jugada dentro de un tiempo determinado, el sistema lo expulsará de la partida.\\
         RF-14 & El sistema ofrecerá un chat que permitirá a los jugadores de una partida comunicarse entre ellos.\\
         \hline
    \end{tabularx}
    \caption{Requisitos funcionales de la aplicación.}
    \label{tab:rf}
\end{longtable}

\begin{table}[hbt]
    \centering
    \begin{tabularx}{\textwidth}{|l|X|}
         \hline
         Código & Descripción \\
         \hline
         RNF-1 & El usuario necesitará acceso a Internet para poder utilizar la aplicación.\\
         RNF-2 & La interfaz de la aplicación estará disponible en castellano.\\
         RNF-3 & Los usuarios deberán registrarse para poder participar en las partidas.\\
         RNF-4 & Un mismo usuario no podrá participar en varias partidas de forma simultánea. \\
         RNF-5 & El usuario podrá continuar una partida en curso independientemente del dispositivo con el que se conecte y el cliente que utilice.\\
         RNF-6 & La interfaz ofrecida estará adaptada a formatos de pantalla horizontales, habituales en los navegadores web, y no ofrecerá capacidades adaptativas para su uso en dispositivos móviles.\\
         \hline
    \end{tabularx}
    \caption{Requisitos no funcionales del sistema.}
    \label{tab:rnf}
\end{table}
\subsection{Diseño del sistema}


%%%%%%%% Entrega 2 y final
%\section{Memoria del proyecto}
%\subsection{Inicio del proyecto}
%\subsection{Ejecución y control del proyecto}
%\subsection{Cierre del proyecto}
%\section{Conclusiones}


\clearpage
\section{Anexos}
%\subsection{Glosario}
%\subsection{Actas de todas las reuniones realizadas}
\subsection{Bibliografía}

\printbibliography
\end{document}
