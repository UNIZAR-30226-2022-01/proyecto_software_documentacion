% Preamble del documento

\documentclass[11pt, a4paper, titlepage]{article}


%%%%%%%%%%%%%%%%%%%%%%%%%%%%%%%%%%%%%%%%%%%%%%%%%%%%%%%%%%%%%%%%%%%%%%%%%%%%%%%%%%%%%%%%%%%%%%%%
% Encoding e idioma Español, 
%%%%%%%%%%%%%%%%%%%%%%%%%%%%%%%%%%%%%%%%%%%%%%%%%%%%%%%%%%%%%%%%%%%%%%%%%%%%%%%%%%%%%%%%%%%%%%%%
\usepackage[utf8]{inputenc}
\usepackage[spanish, es-tabla]{babel} % Para poner cuadro en vez de tabla, etc
%%%%%%%%%%%%%%%%%%%%%%%%%%%%%%%%%%%%%%%%%%%%%%%%%%%%%%%%%%%%%%%%%%%%%%%%%%%%%%%%%%%%%%%%%%%%%%%%



%%%%%%%%%%%%%%%%%%%%%%%%%%%%%%%%%%%%%%%%%%%%%%%%%%%%%%%%%%%%%%%%%%%%%%%%%%%%%%%%%%%%%%%%%%%%%%%%
% Paquetes 
%%%%%%%%%%%%%%%%%%%%%%%%%%%%%%%%%%%%%%%%%%%%%%%%%%%%%%%%%%%%%%%%%%%%%%%%%%%%%%%%%%%%%%%%%%%%%%%%
\usepackage[colorlinks=true, urlcolor=cyan, citecolor=blue, linkcolor=blue, hidelinks]{hyperref} % /url{}, /ref{}...
\usepackage[a4paper,top=2cm,bottom=2cm,left=3cm,right=3cm,marginparwidth=1.75cm]{geometry}
\usepackage{minted} % Código
\usepackage{setspace}
\usepackage{pdfpages}

\usepackage[sorting=none, style=numeric]{biblatex} % aparece 1ro la 1ra citada
\addbibresource{citas.bib}

\usepackage{amsmath}

\usepackage[utf8]{inputenc} 

\usepackage[table]{xcolor} % Colores en tablas/texto/etc

\usepackage{graphicx} % Imágenes

\usepackage{tabularx} % Tablas más potentes
\usepackage{longtable} % Tablas muy largas
\usepackage{multirow}
\usepackage{booktabs}
\usepackage{multicol}
\usepackage{adjustbox} % Ajustar tablas al ancho de la página
\usepackage{pdflscape}

\usepackage{datetime} % Fechas
\usepackage{svg} % para el svg del logo de la eina
\usepackage{titlesec}
\usepackage[section]{placeins}  % para floatbarrier en secciones, 
                                % con [section] se ponen silenciosamente en cada seccion

\usepackage{listliketab}                                
\usepackage{fancyhdr} %pagestyles adicionales

\usepackage{lmodern} % Fuente moderna, \code con negrita
\usepackage{caption} % para poner labels en tabular sin convertirlos en tablas que flotan

\makeatletter
%%%%%%%%%%%%%%%%%%%%%%%%%%%%%%%%%%%%%%%%%%%%%%%%%%%%%%%%%%%%%%%%%%%%%%%%%%%%%%%%%%%%%%%%%%%%%%%%
\renewcommand\paragraph{\@startsection{paragraph}{4}{\z@}%
            {-2.5ex\@plus -1ex \@minus -.25ex}%
            {1.25ex \@plus .25ex}%
            {\normalfont\normalsize\bfseries}} % no sé de donde ha salido esto
\makeatother
\setcounter{secnumdepth}{4} % how many sectioning levels to assign numbers to
\setcounter{tocdepth}{4}    % how many sectioning levels to show in ToC
%\setlength{\parindent}{0pt} % para quitar indentación inicial en párrafos

%%%%%%%%%%%%%%%%%%%%%%%%%%%%%%%%%%%%%%%%%%%%%%%%%%%%%%%%%%%%%%%%%%%%%%%%%%%%%%%%%%%%%%%%%%%%%%%%
% Título para \maketitle
%%%%%%%%%%%%%%%%%%%%%%%%%%%%%%%%%%%%%%%%%%%%%%%%%%%%%%%%%%%%%%%%%%%%%%%%%%%%%%%%%%%%%%%%%%%%%%%%
\title{Plantilla de trabajo}
\author{autor1 \and autor2 \and autor3}

% Macro para mostrar el mes y el año (https://texnique.fr/osqa/questions/1359/commandes-month-year)
\def\monthyear{\ifcase\month\or
  Enero\or Febrero\or Marzo\or Abril\or Mayo\or Junio\or
  Julio\or Agosto\or Septiembre\or Octubre\or Noviembre\or Diciembre\fi
  \space\number\year}

\date{\monthyear}
%%%%%%%%%%%%%%%%%%%%%%%%%%%%%%%%%%%%%%%%%%%%%%%%%%%%%%%%%%%%%%%%%%%%%%%%%%%%%%%%%%%%%%%%%%%%%%%%


%%%%%%%%%%%%%%%%%%%%%%%%%%%%%%%%%%%%%%%%%%%%%%%%%%%%%%%%%%%%%%%%%%%%%%%%%%%%%%%%%%%%%%%%%%%%%%%%
% Estilos
%%%%%%%%%%%%%%%%%%%%%%%%%%%%%%%%%%%%%%%%%%%%%%%%%%%%%%%%%%%%%%%%%%%%%%%%%%%%%%%%%%%%%%%%%%%%%%%%
%\pagestyle{fancy} % con el nombre de la sección y página arriba
%\fancyhf{}
%\pagenumbering{arabic}
%\rfoot{Page \thepage}
\newcommand{\code}{\texttt} % para poner codigo en Monospace


\pagestyle{fancy}
\fancyhf{}
%\rhead{\rightmark}
\lhead{\leftmark}
\rfoot{\thepage}
%%%%%%%%%%%%%%%%%%%%%%%%%%%%%%%%%%%%%%%%%%%%%%%%%%%%%%%%%%%%%%%%%%%%%%%%%%%%%%%%%%%%%%%%%%%%%%%%

% Fin del preamble

% Forza las opciones desde el principio del documento si no se aplican
\AtBeginDocument{
  \urlstyle{same}
  \addtocontents{toc}{\small}
  \addtocontents{lof}{\small}
}
% Cuerpo del documento
\begin{document} 


%%%%%%%%%%%%%%%%%%%%%%%%%%%%%%%%%%%%%%%%%%%%%%%%%%%%%%%%%%%%%%%%%%%%%%%%%%%%%%%%%%%%%%%%%%%%%%%%%		
% PORTADA
%%%%%%%%%%%%%%%%%%%%%%%%%%%%%%%%%%%%%%%%%%%%%%%%%%%%%%%%%%%%%%%%%%%%%%%%%%%%%%%%%%%%%%%%%%%%%%%%
\begin{titlepage}

\thispagestyle{empty}
\includepdf[]{portadaMemoria.pdf}

\end{titlepage}
\newpage
%%%%%%%%%%%%%%%%%%%%%%%%%%%%%%%%%%%%%%%%%%%%%%%%%%%%%%%%%%%%%%%%%%%%%%%%%%%%%%%%%%%%%%%%%%%%%%%%





%%%%%%%%%%%%%%%%%%%%%%%%%%%%%%%%%%%%%%%%%%%%%%%%%%%%%%%%%%%%%%%%%%%%%%%%%%%%%%%%%%%%%%%%%%%%%%%%%		
% CUERPO
%%%%%%%%%%%%%%%%%%%%%%%%%%%%%%%%%%%%%%%%%%%%%%%%%%%%%%%%%%%%%%%%%%%%%%%%%%%%%%%%%%%%%%%%%%%%%%%%
\thispagestyle{empty}
\fontsize{11pt}{11pt}\selectfont

\setcounter{tocdepth}{4}

% Índice con links en negro y no en azul
{
    \hypersetup{linkcolor=black}
    \doublespacing
    \tableofcontents
}

\thispagestyle{empty}

% Texto
\clearpage
\setcounter{page}{1}
\section{Introducción}
A lo largo del documento se va a presentar un proyecto que se ha denominado\textit{ \textbf{World Domination}}. Se trata de un juego de estrategia \textit{on-line} por turnos e inspirado en \textit{Risk}, en el que cada jugador lucha contra el resto por la dominación del mundo. Cada uno de los jugadores cuenta con un número de soldados y territorios controlados por los mismos, así como con una baraja de cartas propia que serán canjeadas por más soldados. En cada turno, los jugadores tendrán que luchar contra tropas bajo el dominio de los rivales, decidiéndose los resultados mediante tiradas de dados. \newline

El juego proporcionará funcionalidades adicionales, que permitirán disfrutar de una experiencia social, fomentando la competición y una mayor participación en futuras partidas. \newline

El sistema debe ser capaz de gestionar las cuentas de los usuarios junto con las partidas que se desarrollen, así como la gestión de turnos y la inicialización y finalización de cada una de las partidas. Algunos de los objetivos fundamentales del sistema son: 

\begin{itemize}
\item Se permitirá a los usuarios registrarse e iniciar sesión con sus credenciales, que constan de un \textit{email} y contraseña. El uso de una cuenta de usuario será necesario para jugar partidas.

\item El sistema contará con una clasificación global de los mejores jugadores y de sus respectivas puntuaciones de las partidas jugadas.

\item Se permitirá a los usuarios personalizar elementos del juego en función de sus preferencias, como las fichas, su foto de perfil, o los dados usados durante las partidas.

\item El sistema recompensará al usuario por haber jugado o ganado un número determinado de partidas con puntos canjeables en la tienda del juego. La tienda será el punto de compra de personalizaciones de los elementos del juego.

\item Se permitirá gestionar una lista de amigos de cada usuario.

\item El sistema permitirá crear partidas públicas y privadas, conforme a las siguientes reglas:
\begin{itemize}
    \item Las partidas serán de 2 a 6 jugadores.
    \item Se permitirá a cualquier usuario entrar a partidas públicas.
    \item En el caso de partidas privadas, solo se permitirá la entrada con una contraseña, introducida al crearla.
    \item Se encontrará disponible un \textit{chat} en la partida.
    \item Se restringirá al usuario a poder jugar solo una partida al mismo tiempo.
\end{itemize}

\item Las partidas se desarrollarán de forma asíncrona, permitiendo a los usuarios responder en cualquier momento. Si el usuario se encuentra conectado, podrá ver una notificación en la pantalla. En el caso contrario, se enviará un correo al mismo notificando que puede realizar la jugada.

\item Habrá un mecanismo de expulsión de jugadores que no respondan o realicen jugadas en un tiempo determinado.

\end{itemize}

A lo largo de la realización del proyecto, será necesario realizar algunas entregas del trabajo llevado a cabo, en reuniones propuestas en unos plazos establecidos:

\begin{itemize}
    \item Una reunión para la evaluación de todas las funcionalidades del juego, con el sistema implementado en una versión final, con el objetivo de realizar ajustes finales, en la semana del 16 al 20 de Mayo.
    \item El sistema final se entregará no más tarde del 1 de junio, con los cambios propuestos durante las reuniones intermedias.
\end{itemize}

A continuación se hablará sobre otros aspectos del proyecto, de la gestión, la organización, análisis de requisitos, así como del diseño del sistema.

\section{Organización del proyecto}
El equipo está formado por 6 integrantes: Samuel García, Guillermo Enguita, Laura Gonzalez, Francisco Crespo, Emilio Estevan y Jorge Grima.
Hemos dividido el proyecto en tres bloques fundamentales:\newline
\begin{itemize}
    \item \textit{Backend} implementado con \textbf{Go} y desarrollo de la API.
    \item Primera versión del \textit{frontend} implementada con \textbf{React}.
    \item Segunda versión del \textit{frontend} implementada con \textbf{Angular}.\newline
\end{itemize}

El trabajo del primer bloque será llevado a cabo por Samuel y Guillermo, el del segundo bloque por Francisco y Emilio, mientras que el del tercer bloque será realizado por Laura y Jorge. 
Aunque cada uno tenga un bloque asignado, en todo momento se ayudará a integrantes encargados del trabajo en otros bloques si fuera necesario. \newline

El equipo decidió nombrar como director a Francisco Crespo, encargado de llevar a cabo la organización principal del proyecto, así como de las entregas necesarias del trabajo.\newline

%%%%%%%%% De 3.2
El responsable del control de configuraciones, construcción del software y aseguramiento de la calidad del producto será Samuel García.


\section{Plan de gestión del proyecto}

A continuación, se describe cómo se llevaran a cabo distintas tareas a realizar durante el proyecto: plan inicial de trabajo, división del trabajo a realizar o control de creación de software, entre otros.

\subsection{Inicio del proyecto}

Diversas tecnologías y lenguajes de programación utilizados en este proyecto eran, en un principio, desconocidos para los integrantes del equipo. Para solventar este problema, hemos tenido que realizar la siguiente formación común:

\begin{itemize}

\item Debido a que toda la documentación necesaria para el proyecto va a ser realizada en \textbf{Latex}, todos los miembros del equipo han adquirido conocimientos básicos para la utilización y explotación de esta herramienta. Para lograr este objetivo, se han realizado dos tutoriales. El primero de ellos <<Learn LaTeX in 30 minutes>>\cite{latextutorial1} se puede encontrar en la página web de \textit{Overleaf}. A su vez, se han seguido los manuales de ayuda proporcionados por la página oficial de \textit{Latex}\cite{latextutorial2}. 

\item Continuando lo anterior, los integrantes del Grupo 1 asistieron a la práctica: \textit{Puesta en marcha de GitHub y fundamentos de Git} impartida por los profesores de la asignatura \textit{Proyecto Software}, la cual consistió en familiarizarse con la herramienta \textbf{Git} y su uso en conjunto con \textbf{GitHub}. A su vez, por medio del tutorial: \textit{Git, GitHub y Publicación Web}\cite{githubtutorial} se habían adquirido los conocimientos básicos de la misma.

\end{itemize}

Los grupos encargados de cada bloque del sistema realizaron formación específica. En primer lugar, los encargados del desarrollo del \textbf{backend} backend del sistema:

\begin{itemize}

\item Harán uso de \textbf{Go} como lenguaje de programación. Se trata de un lenguaje en el que se contaban con unos conocimientos básicos, debido a que ha sido utilizado durante las prácticas de la asignatura \textit{Sistemas Distribuidos}. De cara a fijar conocimientos y a aprender nuevas mecánicas, han realizado el tutorial de la página web oficial de \textbf{Go}. 

\item A su vez, realizaron una formación relacionada con las distintas \textbf{API web}, concretamente el tutorial \textit{Introducción a las API web} de MozillaDeveloper\cite{apisweb}.

\end{itemize}

Por otro lado, el equipo encargado del primer \textit{frontend}, implementado con \textbf{React}: 

\begin{itemize}

\item  Este grupo realizó el tutorial oficial de \textbf{ReactJs}: \textit{Tutorial: Introducción a React}\cite{reactjs}, que tiene una duración estimada de 8 horas. También, realizarán el tutorial de API web anteriormente mencionado\cite{apisweb}.

\end{itemize}

Por último, el equipo encargado de desarrollar el segundo \textit{frontend}, implementado con \textbf{Angular}: 

\begin{itemize}

\item Han realizado el tutorial \textit{Angular y TypeScript}\cite{angular} proporcionado por \textit{DesarrolloWeb}, el cual tiene un tiempo estimado de realización próximo a las 7 horas, debido a que no se ha trabajado con este \textit{framework}

\end{itemize}


\subsection{Control de configuraciones, construcción del software y aseguramiento de la calidad del producto}

Se han establecido los siguientes procedimientos relativos a las tareas de construcción del \textit{software}, gestión de incidencias, automatización, calidad y despliegue del sistema por completo.

\subsubsection{Convenciones y procedimientos para la construcción y control del software}

\begin{itemize}
    \item La organización y nombrado de los ficheros fuente seguirá las prácticas recomendadas por cada uno de los \textit{frameworks} a utilizar. De este modo, se seguirán las siguientes guías de estilo, nombrado de ficheros y organización de proyectos:
    \begin{itemize}
        \item Para \textbf{Angular}, se utilizará la guía de estilo oficial\cite{estiloangular} y la guía de estructura de proyectos propuesta por la documentación oficial\cite{estructuraangular}.

        \item Para \textbf{React}, se utilizará la
        guía de estilo de \textit{airbnb}\cite{estiloreact} y la estructura de proyectos propuesta por la documentación oficial\cite{estructurareact}.

        \item Para \textbf{Go}, se utilizará el estilo impuesto por la herramienta automática de formato de dicho lenguaje\cite{estilogolang} y la estructura de proyectos recomendada por la comunidad\cite{estructuragolang}, compatible con las herramientas de compilación del lenguaje.
    \end{itemize}

    \item La documentación generada durante el desarrollo del proyecto se distinguirá entre \textbf{documentación interna}, dirigida a los desarrolladores, y \textbf{documentación pública}, dirigida a los clientes.\\

    Los ficheros de documentación serán nombrados con minúsculas y podrán contener guiones bajos, sin caracteres especiales. Si forman parte de la documentación del código o interfaces, su nombre empezará por \textbf{documentacion\_}, y si forman parte de guías de uso para los clientes, su nombre empezará por \textbf{guia\_}.

    \item Todo el código será subido a los repositorios de la organización de \textit{Github}\footnote{\href{https://github.com/UNIZAR-30226-2022-01}{\color{blue}{Repositorio de la organización de \textit{Github}}}}, donde todos los miembros tendrán permisos de acceso, modificación y administración de los mismos. Los repositorios serán los siguientes:
    \begin{itemize}
        \item Un repositorio para el \textit{backend} del sistema (\textbf{API})\footnote{\href{https://github.com/UNIZAR-30226-2022-01/proyecto_software_backend}{\color{blue}{Repositorio del \textit{backend} del sistema}}}.

        \item Un repositorio para el \textit{frontend} de \textbf{React}\footnote{\href{https://github.com/UNIZAR-30226-2022-01/proyecto_software_frontend_react}{\color{blue}{Repositorio del frontend de \textit{React}}}}.

        \item Un repositorio para el \textit{frontend} de \textbf{Angular}\footnote{\href{https://github.com/UNIZAR-30226-2022-01/proyecto_software_frontend_angular}{\color{blue}{Repositorio del frontend de \textit{Angular}}}}.

        \item Un repositorio para la documentación\footnote{\href{https://github.com/UNIZAR-30226-2022-01/proyecto_software_documentacion}{\color{blue}{Repositorio de documentación}}}, en el cual se localizarán todos los archivos de documentación interna y de cara al usuario, ya estén terminados como archivos .pdf o en construcción, como archivos de \LaTeX.
    \end{itemize}




\end{itemize}

\subsubsection{Convenciones y procedimientos para el despliegue del software}

\begin{itemize}
    \item Las herramientas necesarias para construir el \textit{software} serán aquellas necesarias para compilar los \textit{frontend} de \textbf{React} y \textbf{Angular} y el \textit{backend} de \textbf{Go} por separado, junto a \textbf{Docker} y \textbf{Docker-compose}. \newline

    Con el objetivo de tener una base de pruebas y desarrollo uniforme, en el repositorio de backend se porporcionarán los siguientes scripts de automatización:
    \begin{itemize}
        \item Dos grupos de scripts de automatización de despliegue de los frontend \textbf{React} y \textbf{Angular} por separado, que crean un contenedor de Docker con el servidor web sirviendo uno de los frontend.

        \item Un grupo de scripts de automatización del \textbf{backend}, que ponen en marcha un par de contenedores ligados mediante \textit{docker-compose} para el servidor de \textit{API} y la base de datos respectivamente.
    \end{itemize}

    Los contenedores de \textit{frontend} serán capaces de comunicarse con los de \textit{backend} mediante ligado de puertos en la máquina host. Durante el desarrollo del sistema, ambos grupos de contenedores se comunicarán en la propia máquina del desarrollador, y para el despliegue se adaptarán tanto las direcciones a utilizar en el código como los puertos en los archivos de despliegue de \textit{Docker}.

    Los \textit{scripts} de automatización se harán cargo de la compilación, obtención de dependencias y puesta en marcha y configuración de cualquier elemento adicional como la base de datos, haciendo posible desplegar un entorno de pruebas común.

    \item El despliegue en entornos de producción se realizará con contenedores de \textbf{Docker} en la infraestructura en la nube de \textbf{Google Cloud}, de tal forma que cada equipo en la nube corresponderá al despliegue en contenedores de cada uno de los \textit{frontend} y del \textit{backend}. Por motivos de seguridad, las claves de acceso a la base de datos o rutas y puertos a utilizar serán establecidas por el cliente e introducidas en los contenedores mediante un fichero de variables de entorno.
\end{itemize}

\subsubsection{Convenciones y procedimientos para el control de la calidad del software}

\begin{itemize}
    \item Se seguirán buenas prácticas para la documentación del código. En concreto, se seguirán las publicadas por \textbf{Google}\cite{documentaciongoogle} para la documentación de la \textit{API} pública, de funciones exportadas de un módulo a otro y de funciones auxiliares que requieran una explicación detallada más allá de la proporcionada por comentarios en el código.

    \item El diseño de la interfaz deberá ser lo más similar posible entre ambos \textit{frontend} para mantener la consistencia, por lo que ambos equipos deberán llegar a un acuerdo al respecto. Además, se seguirán los principios básicos de diseño de interfaces y usabilidad, como las heurísticas de Nielsen\cite{heuristicasnielsen}.

    \item El código de cada una de las partes del proyecto deberá ser revisado por ambos miembros encargados de él, tanto de forma manual como automática, tras implementar nuevas funcionalidades. Para  ello, se establece la siguiente estrategia:
    \begin{itemize}
        \item El \textit{backend} proporcionará un conjunto de pruebas automáticas sobre cada uno de los elementos de la \textit{API}, ya que el lenguaje elegido, \textit{Go}, proporciona de forma nativa herramientas de test automático y la posibilidad de hacer de cliente de la \textit{API} sin necesitar de intermediario al \textit{frontend}.

        La adición de una nueva funcionalidad a la \textit{API} deberá garantizar que tanto su prueba automática como el resto de ellas no muestran ningún tipo de error.

        \item El \textit{frontend} realizará pruebas exploratorias a la hora de implementar funcionalidades que involucren llamar a la \textit{API}, teniendo la garantía de que ha pasado las pruebas automáticas previamente.
        \end{itemize}


    \item Las entregas a los clientes serán probadas íntegramente por todos los miembros del equipo en conjunto para garantizar que no existen problemas en las funcionalidades implementadas hasta dicho momento.

    La estrategia de pruebas previas a la entrega será la siguiente:
    \begin{itemize}
        \item Se acordará, mediante una reunión de al menos un integrante de cada par de miembros de cada equipo, un guión de prueba exploratorias común para ambos \textit{frontend}, que abarque todas las funcionalidades del juego implementadas hasta el momento por ambos.

        \item Cada uno de los miembros de cada \textit{frontend} realizará dichas pruebas siguiendo el guión sobre su \textit{frontend}, mientras que cada miembro del equipo de \textit{backend} las realizará sobre uno de ellos respectivamente, comprobando que las interacciones desde el lado del \textit{backend} son correctas.

        Una vez realizadas las pruebas, los resultados se comunicarán al responsable de estos aspectos, que coordinará la corrección de los errores observados y repetirá el proceso de prueba si es necesario.
    \end{itemize}
\end{itemize}

\subsection{Calendario del proyecto, división del trabajo, coordinación, comunicaciones, monitorización y seguimiento}

%%%%%%%%%%%%%%%%%%%%%%%%%%%%%%%
%%%%%%%%%%%%%%%%%%%%%%%%%%%%%%%
%%%%%%%%%%%%%%%%%%%%%%%%%%%%%%%
%%%%%%%%%%%%%%%%%%%%%%%%%%%%%%%%%%%%%%%%%%%%%%%%%%%%%%%%%%%
% De control de configuraciones y software, a incluir en comunicaciones:
%%%%%%%%%%%%%%%%%%%%%%%%%%%%%%%%%%%%%%%%%%%%%%%%%%%%%%%%%%%

\item La gestión de incidencias y tareas pendientes se realizará mediante el gestor de incidencias de \textbf{GitHub} del repositorio relacionado con ellas. Debido a que no permite crear incidencias que abarquen múltiples repositorios, cada uno de ellos tendrá las incidencias globales que afecten a varios de ellos replicadas, con las tareas que les corresponda a cada uno.

    \item Todas las tareas, funcionalidades a implementar o ampliar y errores a arreglar serán documentados en incidencias individuales con una descripción concisa. Estas formarán parte de hitos para entregas y versiones intermedias si aplica. Las incidencias se clasificarán utilizando etiquetas. De la misma forma que con las incidencias, los hitos globales estarán replicados en cada repositorio con las tareas e incidencias que correspondan a él.

    Las incidencias serán asignadas a quienes trabajen en ella para mantener un control de actividades y tiempo, y podrán ser cerradas y abiertas de nuevo libremente.

    \item Todos los \textit{commits} a realizar no requerirán aprobación, pero deberán ser probados para que no causen problemas de compilación, y deberían ser tan funcionalmente completos como sea posible. Además, no deberán abarcar un número excesivo de cambios en el código para favorecer que se puedan deshacer si es necesario, y deberán tener una descripción de los cambios realizados concisa.

    Por ello, se han establecido los siguientes objetivos a comprobar cada vez que se realiza un commit:
    \begin{itemize}
        \item Se comprobará que el código compila tras implementar dicho commit, y que todos los test automáticos implementados hasta el momento siguen siendo correctos.
        \item El ámbito del commit no deberá abarcar más allá del grupo de tareas de una incidencia en concreto a no ser que no sea posible evitarlo. Esto ayuda a evitar realizar un número excesivo de cambios.
        \item Los commits no deberá exceder de 500 nuevas líneas siempre que sea posible, excluyendo la subida de ficheros o implementaciones de funciones inherentemente largas, como los test automáticos.
    \end{itemize}


%%%%%%%%%%%%%%%%%%%%%%%%%%%%%%%
%%%%%%%%%%%%%%%%%%%%%%%%%%%%%%%
%%%%%%%%%%%%%%%%%%%%%%%%%%%%%%%
%%%%%%%%%%%%%%%%%%%%%%%%%%%%%%%
%%%%%%%%%%%%%%%%%%%%%%%%%%%%%%%
%%%%%%%%%%%%%%%%%%%%%%%%%%%%%%%

La coordinación y seguimiento de objetivos del proyecto se va a seguir mediante un diagrama de \textit{Gantt}, apreciable en la Figura 1, el cual podrá ser modificado en el futuro debido a posibles adelantos o retrasos en alguna de las tareas, y también por tareas imprevistas que pueden surgir durante la realización del proyecto.\\

\begin{landscape}
    \pagestyle{empty}
    \begin{figure}[!p]
    \centering
    \includegraphics[scale=0.6]{diagramas/gant.PNG}
    \caption{Diagrama de Gantt.}
    \label{fig:my_label}
\end{figure}
\end{landscape}

Se han establecido las siguiente divisiones de trabajo junto a las personas asignadas a cada parte:

\begin{itemize}
    \item \textbf{Backend}: Guillermo Enguita y Samuel García
    \item \textbf{Frontend}
    \begin{itemize} 
        \item \textbf{Angular}: Laura González y Jorge Grima
        \item \textbf{React}: Fran Crespo y Emilio Estevan
    \end{itemize}
\end{itemize}

La división de tareas diarias internas en cada uno de los equipos se consensuará entre sus miembros, y se preferirá en todo momento la programación por pares.\\

Tanto el \textit{frontend} como el \textit{backend} poseerán un calendario de seguimiento propio en el cual se establecen los objetivos a realizar junto a sus fechas limite para ser completadas. \\

Ambos tienen en común dos fechas las son \textbf{8 de abril} debido a que esta es la es la entrega intermedia, en donde se le mostrará al cliente una primera versión funcional del proyecto
y \textbf{30 de junio} porque esta es la fecha de la entrega del proyecto.\\

Cada dos semanas todos los miembros del equipo rellenarán un cuestionario indicando las horas trabajadas y el progreso realizado. Por otro lado las actas se tomarán por turnos durante las reuniones con el cliente en las fechas indicadas por él mismo.\\

Las métricas de rendimiento del equipo irán marcadas por el estado por cada una de las incidencias y tareas pendientes de cada uno de los repositorios. Si se diese el caso de que algún equipo quedase rezagado respecto al resto o sufriera algún imprevisto, éste deberá comentárselo al coordinador del equipo para que así se pueda tomar una decisión sobre cómo actuar al respecto. 

\subsubsection{Calendario de seguimiento para el frontend}
El calendario interno para el \textit{frontend} va a ser el mismo para ambas implementaciones. La tabla \ref{table:frontend} muestra las tareas a realizar por este equipo. \newline


\renewcommand{\arraystretch}{1.3}
\begin{table}[hbt!]
\begin{adjustbox}{minipage=18cm, center}
\begin{tabularx}{\textwidth}{|c|X|c| }
\hline
Entrega & Tarea & Fecha objetivo\\ \hline
\multirow{9}{*}{Entrega 1} & Diseño inicial y \textit{Wireframing} de las pantallas & 18 marzo \\\cline{2-3}
& Pantalla inicial & 18 marzo\\\cline{2-3}
& Pantalla de registro de usuario & 18 marzo \\\cline{2-3}
& Pantalla de inicio de sesión &  18 marzo \\\cline{2-3}
& Pantalla del perfil de usuario &  25 marzo \\\cline{2-3}
& Pantalla de búsqueda de partida &  25 marzo \\\cline{2-3}
& Pantalla de configuración de partida & 25 marzo \\\cline{2-3}
& Pantalla del buzón de notificaciones del usuario &  8 abril \\\cline{2-3}
& Implementación de una versión inicial de la pantalla del juego y del mapa & 8 abril
\\\hline
\multirow{6}{*}{Entrega 2} & Pantalla de fin de partida & 15 abril \\\cline{2-3}
& Implementación de una versión final de la pantalla del juego y del mapa & 30 mayo\\\cline{2-3}
& Pantalla de la tienda & 29 abril \\\cline{2-3}
& Pantalla de personalización de usuario & 6 mayo\\\cline{2-3}
& Pantalla del \textit{chat} en la partida & 20 mayo \\\cline{2-3}
& Pantalla de la clasificación global de usuarios& 27 mayo \\\cline{2-3}
\hline
\end{tabularx}
\caption{Tabla de seguimiento del \textit{frontend}}
\label{table:frontend}
\end{adjustbox}
\end{table}
%\end{center}
\FloatBarrier    

\newpage

\subsubsection{Calendario de seguimiento para el backend}
 La tabla \ref{table:backend} muestra las tareas de este equipo a realizar. \newline

\renewcommand{\arraystretch}{1.3}
\begin{table}[hbt!]
\begin{adjustbox}{minipage=18cm, center}
\begin{tabularx}{\textwidth}{|c|X|c| }
\hline
Entrega & Tarea & Fecha objetivo \\\hline
\multirow{6}{*}{Entrega 1}& Diseño el modelo de datos del sistema & 18 marzo\\\cline{2-3}
 & Implementación de la base de datos & 18 marzo \\\cline{2-3}
 & Implementación de los mecanismos de autenticación de usuarios frente al sistema y la persistencia de sesiones & 25 marzo \\\cline{2-3}
 & \textit{API} de búsqueda y configuración de partidas & 25 marzo\\\cline{2-3}
 & \textit{API} de notificaciones &  1 abril \\\cline{2-3}
 & Versión parcial de la \textit{API} de lógica del juego & 8 abril \\\cline{2-3}
\hline
\multirow{5}{*}{Entrega 2} & Versión final de la \textit{API} de lógica del juego & 20 mayo \\\cline{2-3}
& \textit{API} de la tienda & 29 abril \\\cline{2-3}
& \textit{API} de personalización de usuario& 6 mayo \\\cline{2-3}
& Sistema de \textit{chat} & 20 mayo \\\cline{2-3}
& \textit{API} de la clasificación de usuarios &  27 mayo \\\cline{2-3}
\hline
\end{tabularx}
\caption{Tabla de seguimiento del \textit{backend}}
\label{table:backend}
\end{adjustbox}
\end{table}
\FloatBarrier
 
La responsable del seguimiento de todos estos aspectos será Laura González.

\section{Análisis y diseño del sistema}

\subsection{Análisis de requisitos}
Los requisitos funcionales de nuestra aplicación se pueden encontrar detallados en la tabla \ref{tab:rf}. La tabla \ref{tab:rnf} recoge los requisitos no funcionales. Ambos tipos de requisitos se encuentran claramente detallados e identificados, para facilitar su referencia en documentación futura.
\renewcommand{\arraystretch}{1.3}\\
\begin{longtable}[h!]{| p{.10\textwidth} | p{.90\textwidth} |} 
    \caption{Requisitos funcionales de la aplicación.}
    \label{tab:rf}
    \centering
        \hline
         Código & Descripción  \\
         \hline
         RF-1 & El sistema permitirá a los usuarios registrarse, utilizando un correo electrónico y estableciendo un nombre de usuario y una contraseña.\\
         \hline
         RF-2 & El sistema permitirá a los usuarios iniciar sesión, utilizando su nombre de usuario y su contraseña.\\
         \hline
         RF-3 & El sistema proporcionará a los usuarios un juego de \textit{Risk}.\\
         \hline
         RF-3.1 & Al comenzar la partida, se distribuirá un número de ejércitos a cada jugador y por turnos los irán colocando en el mapa.\\
         \hline
         RF-3.2 & Opcionalmente, la disposición inicial de las tropas mencionada en el \textbf{RF-3.1}  se podrá hacer de forma aleatoria.\\
         \hline
         RF-3.3 & Cada turno de juego dispondrá de las siguiente fases: Refuerzo, Ataque y Fortificación. \\
         \hline
         RF-3.4 & Al empezar el turno del jugador, se le otorgará un número de ejércitos en función del número de territorios y continentes que ocupe.\\
         \hline
         RF-3.5 & Durante la fase de refuerzo, el jugador deberá colocar todas sus tropas distribuyéndolas como desee entre los territorios que domine.\\
         \hline
         RF-3.6 & El jugador podrá cambiar un conjunto de 3 cartas para recibir ejércitos durante la fase de refuerzo.\\
         \hline
         RF-3.7 & El jugador recibirá una carta al final del turno, si durante dicho turno ha conquistado algún territorio.\\
         \hline
         RF-3.8 & Las cartas pueden ser de 4 tipos: infantería, caballería, artillería y comodines.
         Además, cada carta tendrá un territorio asignado (excepto los comodines).\\
         \hline
         RF-3.9 & Para que un cambio de cartas sea válido, se deberán cambiar 3 cartas del mismo tipo, dos del mismo tipo junto a un comodín o una carta de cada tipo. \\
         \hline
         RF-3.10 & El número de ejércitos recibidos por cada cambio dependerá del número de cambios realizados. El primer cambio otorgará 4 ejércitos y los cambios consecutivos darán 2 ejércitos más que el anterior. El sexto cambio dará un total de 15 ejércitos, y a partir de éste el número de ejércitos se incrementará en 5 por cada cambio. \\
         \hline
         RF-3.11 & El jugador estará obligado a cambiar cartas siempre que tenga 5 o más cartas en su mano. \\
         \hline
         RF-3.12 & El jugador recibirá 2 ejércitos extra si alguna de las cartas que cambia corresponde a un territorio que ocupe. Esos 2 ejércitos se añadirán a dicho territorio. En caso de que esto ocurra con más de una carta. \\
         \hline
         RF-3.13 & En caso de que se de la situación mencionada en el \textbf{RF-3.12} con más de una carta del cambio, el jugador podrá elegir a cual de los territorios se asignan las tropas adicionales. \\
         \hline
         RF-3.14 & Durante la fase de ataque el jugador podrá atacar territorios adyacentes los suyos, siempre y cuando tenga por lo menos dos ejércitos en la región desde la que ataca.\\
         \hline
         RF-3.15 & Al atacar, tanto atacante como defensor lanzarán los dados. El atacante podrá lanzar entre 1 y 3 dados, necesitando por lo menos un ejército más que el número de dados. Por otro lado, el defensor lanzará 2 dados si su territorio tiene 2 o más ejércitos y 1 dado en caso contrario.\\
         \hline
         RF-3.16 & En caso de que el dado más alto lanzado por el atacante sea de valor mayor que el del defensor, el territorio defensor perderá un ejército. Si el valor fuera igual o menor, el territorio atacante perderá un ejército. Si ambos han lanzado dos o más dados, se realizará el mismo proceso con el segundo dado más alto de cada uno.\\
         \hline
         RF-3.17 & Si tras un ataque, el territorio defensor queda sin ejércitos, el atacante deberá mover a dicha región tantas tropas como dados haya tirado durante el último ataque menos el número de ejércitos que haya perdido en el ataque. \\
         \hline
         RF-3.18 & Al final del turno, el jugador tendrá la posibilidad de fortificar un territorio, moviendo tropas de un territorio a otro. Para ello, ambos territorios deberán pertenecer al jugador y estar conectados por un camino de regiones ocupadas por él. Cabe destacar que no se podrá dejar ningún territorio sin tropas.\\
         \hline
         RF-4 & El sistema ofrecerá a los usuarios un menú que permitirá buscar partidas públicas en curso, destacando las partidas en las que participen sus amigos.\\
         \hline
         RF-5 & El sistema recompensará a los jugadores al ganar partidas, otorgándoles puntos canjeables.\\
         \hline
         RF-6 & El sistema ofrecerá una tienda a los usuarios en la que podrán utilizar los puntos descritos en el requisito \textbf{RF-5} para comprar diferentes cosméticos que permitan la personalización. \\
         \hline
         RF-7 & El sistema permitirá a los usuarios personalizar elementos del juego.\\
         \hline
         RF-7.1 & El sistema permitirá a los usuarios modificar su foto de perfil y su biografía.\\
         \hline
         RF-7.2 & El sistema permitirá a los usuarios personalizar sus dados y fichas de juego, utilizando los cosméticos comprados en la tienda definida en el requisito \textbf{RF-6}.\\
         \hline
         RF-7.3 & Las personalizaciones realizadas serán visibles por todos los usuarios.\\
         \hline
         RF-8 & El sistema permitirá a los usuarios gestionar una lista de amigos.\\
         \hline
         RF-9 & El sistema permitirá a los usuarios consultar una clasificación global, la cuál mostrará a los mejores jugadores junto con sus respectivas puntuaciones.\\
         \hline
         RF-10 & El sistema permitirá a los usuarios consultar el perfil personal de otros jugadores.\\
         \hline
         RF-11 & El sistema permitirá a los usuarios crear una partida.\\
         \hline
         RF-11.1 & El sistema permitirá crear partidas públicas, a las que cualquier usuario se podrá unir.\\
         \hline
         RF-11.2 & El sistema permitirá crear partidas privadas, de forma que el usuario deberá definir una contraseña para el acceso a dicha partida.\\
         \hline
         RF-11.3 & El sistema permitirá al usuario establecer el número de jugadores de la partida (de 2 a 6 jugadores).\\
         \hline
         RF-12 & Las partidas se desarrollarán de forma asíncrona.\\
         \hline
         RF-12.1 & Cuando llegue el turno del jugador, el sistema le avisará mostrando una notificación en la pantalla si está conectado a la aplicación.\\
         \hline
         RF-12.2 & Si el usuario no está conectado, el sistema enviará el aviso a través de correo electrónico.\\
         \hline
         RF-12.3 & Si el usuario no realiza su jugada dentro de un tiempo determinado, el sistema lo expulsará de la partida.\\
         \hline
         RF-13 & El sistema ofrecerá un \textit{chat} que permitirá a los jugadores de una partida comunicarse entre ellos.\\
         \hline
\end{longtable}

\newpage
\begin{table}[h!]
    \centering
    \begin{tabularx}{\textwidth}{|l|X|}
         \hline
         Código & Descripción \\
         \hline
         RNF-1 & El usuario necesitará acceso a \textit{Internet} para poder utilizar la aplicación.\\
         \hline
         RNF-2 & La interfaz de la aplicación estará disponible en castellano.\\
         \hline
         RNF-3 & Los usuarios deberán registrarse para poder participar en las partidas.\\
         \hline
         RNF-4 & Un mismo usuario no podrá participar en varias partidas de forma simultánea. \\
         \hline
         RNF-5 & El usuario podrá continuar una partida en curso independientemente del dispositivo con el que se conecte y el cliente que utilice.\\
         \hline
         RNF-6 & La interfaz ofrecida estará adaptada a formatos de pantalla horizontales, habituales en los navegadores \textit{web}, y no ofrecerá capacidades adaptativas para su uso en dispositivos móviles.\\
         \hline
    \end{tabularx}
    \caption{Requisitos no funcionales del sistema.}
    \label{tab:rnf}
\end{table}

\newpage

\subsection{Diseño del sistema}

El diseño y planificación de los elementos y lógica de un sistema es fundamental para su desarrollo. En consecuencia, se han desarrollado los correspondientes diagramas arquitecturales que reflejan los componentes de la aplicación, así como la interacción entre ellos. \newline

En primer lugar, se ha modelado el \textbf{diagrama de componentes}, apreciable en la Figura 2, el cual representa los componentes principales en ejecución, sus interfaces y conectores que les permiten interactuar. Concretamente, se puede apreciar los dos componentes cliente, uno para cada tecnología a utilizar, los cuales implementan la lógica de presentación y contienen parte de lógica interna de la aplicación, y se comunican con el servidor mediante la misma interfaz (\textit{API}). En siguiente lugar, se encuentra el servidor \textit{web}, el cual ofrece una interfaz común a los clientes e implementa la lógica de negocio y el acceso a los datos del sistema. Por último, se encuentra el gestor de base de datos, el cual aporta la persistencia necesaria al sistema.

\begin{figure}[h!]
    \centering
    \includegraphics[width=1\linewidth]{diagramas/DC.png}
    \caption{Diagrama de componentes.}
\end{figure}


Por otro lado, en la Figura 3 se ilustra el \textbf{diagrama de despliegue}, donde se puede apreciar la distribución de los distintos nodos que componen el sistema. En consecuencia, se ha representado el servidor \textit{web} junto con los artefactos de comprenden la lógica interna del servidor y el sistema gestor de base de datos. Este nodo da soporte a los dos tipos de clientes posibles, uno con cada tecnología, los cuales implementan cierta lógica interna de la aplicación y se encargan de mostrar y gestionar la interfaz al usuario. 

\begin{figure}[h!]
    \centering
    \includegraphics[width=1\linewidth]{diagramas/DD.png}
    \caption{Diagrama de despliegue.}
\end{figure}

Dadas las características del tipo de aplicación \textit{web}, se va a desarrollar una \textbf{arquitectura} cliente-servidor como modelo de diseño, en la que los clientes lanzarán peticiones al servidor, este las gestionará y mantendrá el estado del sistema y su persistencia, y devolverá la respuesta correspondiente. Además, como \textbf{protocolo de comunicación} entre los nodos de la red, se utilizará \textbf{HTTP}.

\subsubsection{Tecnologías elegidas}

En cuanto a las \textbf{tecnologías elegidas}, cabe destacar el uso de \textbf{HTML}, \textbf{CSS} y \textbf{JavaScript} para la parte del frontend de la aplicación, en la que se distinguirá la utilización de dos librerías de creación de interfaces de usuario: \textbf{React} y \textbf{Angular}. Por ello, la aplicación divergirá en dos clientes ligeramente distintos para el usuario e internamente construidos de distinta forma, que se comunicarán con el servidor web central, el cual no distinguirá entre tipos de clientes, sino que se limitará a atender peticiones, gestionarlas y devolver la respuesta adecuada. \newline

Por otro lado, la tecnología elegida para el servidor o backend es el lenguaje \textbf{Go}, ya que es uno de los lenguajes más potentes y con más futuro en sistemas distribuidos, el cual ofrece infinidad de herramientas de gestión para un servidor de las características del sistema. Para la persistencia de datos del sistema, se recurrirá al gestor \textbf{PostgreSQL}, ya que cubre con garantías todas las necesidades de la aplicación. \newline


Para el backend se ha valorado utilizar las siguientes librerías auxiliares para servir contenidos \textbf{HTTP}:

\begin{itemize}
    \item \textit{Frameworks de web}: Se ha valorado utilizar frameworks web como \textbf{Echo} o \textbf{Gin}. Sin embargo, se ha considerado que  la documentación era escasa en todos los \textit{frameworks} evaluados, y su uso forzaría a una dependencia en las mismas. 
    
    \item Por otro lado, se ha optado por utilizar una librería auxiliar a la librería estándar de \textbf{HTTP} de \textbf{Go}, \textbf{Chi}, que se caracteriza por ser solamente una ampliación a la librería estándar muy sencilla de utilizar, y que permite establecer diferentes \textit{middleware} programables para diferentes \textit{URLs}, algo que se va a necesitar.
    
    \item Librerías auxiliares de \textbf{SQL}: Se ha evaluado utilizar \textbf{ORMs} como \textbf{Gorm}, pero se ha considerado que el modelo de datos es lo suficientemente sencillo como para utilizar la librería \textbf{SQL} y consultas tradicionales.
\end{itemize}

\subsubsection{Aspectos a considerar}

Entre otros \textbf{aspectos a considerar}, cabe destacar los siguientes. El despliegue e instalación de la aplicación se llevará a cabo mediante contenedores de \textbf{Docker}, preferiblemente en un sistema operativo \textbf{Linux}, dada la facilidad de uso y el rápido despliegue que ofrece. Por otro lado, el modelo de datos a implementar se limita al estándar \textbf{SQL}, es decir, un modelo relacional. El servidor ofrece una \textit{API Web} de tipo \textbf{REST}, de forma que cada petición contenga toda la información necesaria para comprenderla, sin que exista un estado explícito a consensuar entre cliente y servidor. \newline

Por último, se barajaron otras opciones a las comentadas anteriormente:

\begin{itemize}
    \item Utilizar otro lenguaje para el \textit{backend}, como \textbf{Java}. Sin embargo, se decidió que era más apropiado el uso de \textbf{Go}, que es uno de los lenguajes para sistemas distribuidos más usados, ya que ofrece múltiples recursos para soportar distintos protocolos de comunicación, junto con aspectos de seguridad, así como simplicidad, compatibilidad y eficiencia en la gestión de peticiones de red dentro de su propia librería estándar. Además, el equipo está habituado al desarrollo de aplicaciones web con \textbf{Go}, hecho que impulsa su utilización.
    
    \item  Manejar otro sistema gestor de base de datos similar, como un gestor de bases de datos \textbf{NoSQL}. Actualmente existe un amplio abanico de opciones para elegir y como las necesidades de la aplicación se cubren con prácticamente cualquier gestor, se determinó el uso de  \textbf{PostgreSQL} ya que es uno de los gestores de código abierto más utilizados y que ofrece aspectos muy interesantes. También, el equipo está habituado al uso de este gestor en concreto.
    
    \item Existen muchas librerías para el desarrollo de interfaces \textit{web}. Sin embargo, se decidió utilizar  \textbf{React} y \textbf{Angular} porque ofrecen una gran cantidad de recursos que permiten construir elementos complejos mediante instrucciones muy simples, así como el soporte y documentación que tienen dado su extendido uso en la actualidad.
\end{itemize}


% Entrega 2 y final
\section{Memoria del proyecto}
introducción

\subsection{Inicio del proyecto}
%Apartado react
En relación al equipo del frontend de \textbf{React} se realizo de forma correcta el plan establecido. Realizaron el tutorial correspondiente y, posteriormente, realizaron un proyecto de prueba de cara a probar y poner en práctica las competencias adquiridas. También, investigaron y probaron diversos paquetes o librerías adicionales de \textit{React}, como aprender a realizar las conexiones con la API y gestionar cookies. \\
Cabe resaltar que \textbf{se cumplieron los objetivos} deseados de acuerdo a la planificación inicial. \\

% Apartado backend
Con lo que respecta al equipo de \texit{backend}, la formación inicial necesaria no fue tan pesada como en el caso del \texit{frontend}, ya que ambos componentes del equipo tenían experiencia con las tecnologías utilizadas. Hemos utilizado \textit{Go} recientemente para las prácticas de la asignatura \textit{Sistemas Distribuidos} y \texit{PostgreSQL} como gestor de bases de datos en \textit{Sistemas de la Información}.

\subsection{Ejecución y control del proyecto}

\subsubsection{Frontend Angular}

\subsubsection{Frontend React}

%% Repartición de tareas

El trabajo desarollado por el equipo de \textit{React} ha contado con una muy buena comunicación y coordinación por parte de sus integrantes debido, en gran medida, a haber trabajado juntos en múltiples ocasiones. Esta situación ha propiciado que el entorno de trabajo haya sido cordial y dinámico. \\

La coordinación y el modelo de trabajo han seguido las siguientes indicaciones:

\begin{itemize}
    \item Las decisiones y partes de desarrollo de mayor envergadura o complejidad han sido desarrolladas concurrentemente por los dos integrantes. 
    \item Por otro lado, en el resto de ejecuciones se ha optado por realizar dos reuniones semanales de cara a poner en común aspectos realizados, fijar distintos objetivos a lograr o especificar diseños de pantallas.
    \item Por último, se han repartido tareas individuales de forma equitativa, no excesivamente extensas.
\end{itemize}

Como se ha mencionado anteriormente, el \textbf{control de trabajo} de la pareja lo han ido produciendo en las múltiples reuniones realizadas. Por otro lado, se han utilizado los \textit{issues de Git}, de cara a mostrar las tareas realizadas y las pendientes de realizar. \\

Respecto al seguimiento de los procedimientos y herramientas establecidas en un inicio, cabe destacar:

\begin{itemize}
    \item La tecnología elegida inicialmente ha sido adecuada para el desarrollo del sistema, y ha facilitado en gran medida la implementación del mismo debido, en gran medida, a la gran cantidad de información y tutoriales que se puede encontrar por la web relacionada con \textit{React}.

    \item Cabe destacar que se han añadido o modificado múltiples librerias y paquetes con respecto a los establecidos al inicio del proyecto. Esto se ha debido a que se han ido descubriendo progresivamente ciertos paquetes que, en un principio, no se contaba con utilizar. Los más destacables son:
        \begin{itemize}
            \item \textit{React-router-dom}, de cara a navegar entre las distintas pantallas. \footnote{\href{https://v5.reactrouter.com/web/guides/quick-start}{\color{blue}{\textit{ Página oficial react-router-dom.}}}}
            \item \textit{SweetAlert2}, librería de pantallas auxiliares.
            \footnote{\href{https://sweetalert2.github.io}{\color{blue}{\textit{ Página oficial SweetAlert2.}}}}
            \item \textit{Universal Cookies}, gestión, creación y explotación de cookies web.
            \footnote{\href{https://www.npmjs.com/package/universal-cookie}{\color{blue}{\textit{ Página oficial universal-cookies.}}}}
        \end{itemize}
\end{itemize}

Por otro lado, con respecto al \textbf{calendario de tareas} fijado al inicio, no se han cumplido todos los objetivos planeados en un inicio. El motivo reside  en que se ha optado por realizar variaciones y priorizar ciertas tareas.
Debido a esto, las pantallas de \textit{Perfil de usuario}, \textit{Búsqueda de partida} y \textit{Buzón de notificaciones} no han sido realizadas para la primera entrega y se ha puesto un mayor enfasis en la implementación del \textit{Mapa de juego}. 
Pese a las variaciones comentadas anteriormente, el equipo considera que se han cumplido los objetivos de esta entrega intermedia debido a que la pantalla del \textit{Mapa} se trata de la más compleja de implementar y desarrollar y tiene una importancia relevante sobre todas las demás. \\

Por último, en relación al despliegue del frontend web cabe destacar un problema que surgió, debido a que existen incompatibilidades entre \textit{macOS} y los archivos iniciales de\textit{Docker}. Tras revisar y estudiar el error en prufundidad el equipo consiguió solucionar el problema. 
\subsubsection{Backend}

El proceso de ejecución y control del proyecto para el equipo de \textit{backend} se ha considerado satisfactorio.\\

%% Repartición de tareas

En términos de repartición de trabajo, el comienzo del proceso de desarrollo ha consistido en programación en pareja de los aspectos fundamentales del sistema, como la definición del modelo de datos, o la de módulos base, funciones críticas y estructuras a utilizar en el servidor.

Una vez realizado esto, se han repartido tareas individuales equitativamente y puesto en común en reuniones más breves todas aquellas funcionalidades comunes a todas las tareas a realizar.

%% Tecnologías / librerías elegidas
%% Seguimiento de nombrado de ficheros y formato, etc
%% Seguimiento de control de configuraciones
%% seguimiento de asignación de tareas, issues, etc.
%% control de calidad
%%      tests realizados
%%      control de calidad en commits
%% despliegue

Respecto al seguimiento de los procedimientos establecidos, se han seguido los siguientes aspectos con éxito:

\begin{itemize}
    \item Las tecnologías y librerías elegidas inicialmente han sido adecuadas para el desarrollo del sistema, y han facilitado en gran medida la implementación del mismo, gracias a la libería estándar de \textit{Go} y el framework \textit{Chi}.

    \item El seguimiento de las reglas de formato y documentación de código se han seguido fácilmente debido a las herramientas proporcionadas por \textit{Go}, como \textit{godoc} o \textit{gofmt}.

    \item Se ha realizado un seguimiento y asignación de las tareas a realizar mediante incidencias de \textit{Github},  listas de tareas en cada uno de ellos e hitos, tal y como se había acordado.

    \item Se han realizado test automáticos para cada funcionalidad base implementada, que corresponde a un conjunto de llamadas a la \textit{API} relacionadas con un mismo aspecto del sistema, como la gestión de amigos o la implementación de una fase del juego.

    Así mismo, se ha conseguido fácilmente que dichos test se comporten como un cliente del sistema gracias a la librería estándar de \textit{Go}.

    \item Se ha vigilado en todo momento que los commits fueran correctos tal y como ha sido establecido, y que se completaran los test automáticos con éxito antes de subirlos.

    \item Se ha conseguido automatizar un despliegue completo de cada uno de los \textit{frontend} y del servidor de la \textit{API} para facilitar las pruebas exploratorias y de comportamiento del sistema fuera de los test automáticos.
\end{itemize}

% discutir como se ha seguido el calendario

Por otro lado, se ha conseguido seguir el calendario de tareas establecido en todo momento, llegando a implementar una versión inicial del juego, consistente en el inicio y la gestión de las partidas y la primera fase de un turno, así como todos los aspectos fundamentales del diseño del sistema (el modelo de datos, autenticación de usuarios o una versión preliminar de toda la \textit{API}, por ejemplo) y funcionalidades adicionales como los aspectos sociales del juego.

La única divergencia respecto a la planificación inicial ha sido el aplazamiento de la implementación de la \textit{API} de notificaciones una semana, aún dentro del límite, debido a que se ha detectado que dependía de funcionalidades de la lógica de juego.

Por último, durante el desarrollo del sistema se han detectado los siguientes problemas respecto al control de versiones, construcción y despliegue del software:
\begin{itemize}
    \item Al realizar pruebas exploratorias de los \textit{frontend} con el despliegue automático de contenedores de \texit{Docker}, se han experimentado problemas no observados al realizar los test automáticos debido a que los servidores de \textit{frontend} y \textit{backend} se encontraban en dominios diferentes.

    Esto ha requerido realizar cortas reuniones con ambos equipos e implementar un middleware de \textit{CORS} \footnote{\href{https://developer.mozilla.org/en-US/docs/Web/HTTP/CORS}{\color{blue}{\textit{ Información sobre CORS por Mozilla.}}}}. Adicionalmente, esto ha provocado problemas inesperados con el tratamiento de \textit{cookies} que también ha requerido reuniones con ambos equipos.

    \item El despliegue automático ha requerido ligeros cambios para que fuera posible su ejecución en equipos con \textit{macOS}.
\end{itemize}
% TODO hablar de pruebas

\subsection{Cierre del proyecto}
% Apartado Backend
En general, el tiempo utilizado para la realización de las distintas tareas del \textit{backend} se ajusta razonablemente a la estimación inicial.
Aún así, a la hora de realizar el proyecto hemos podido comprobar que las tareas elegidas para las estimaciones eran demasiado generales.\\

Principalmente, englobar la lógica del juego como una sola tarea ha dificultado una estimación temporal más precisa. En la práctica, dicha tarea fue dividida en distintas partes, principalmente el diseño de la máquina de estados, la implementación del mapa, las diferentes fases de juego, ...\\
Creemos que estas divisiones se podrían haber realizado al inicio del proyecto, analizando más exhaustivamente el funcionamiento del juego para permitir estimaciones más precisas. \\

Mencionamos también que aunque no se hubiera tenido en cuenta en la estimación inicial, el tiempo empleado en la implementación de las pruebas fue muy elevado y habría sido conveniente haberlo tenido en cuenta durante la planificación de tareas. \\  

Destacamos que si el desarrollo del \textit{backend} sigue progresando al ritmo actual, se acabará su implementación antes de lo planeado en el calendario. De esta forma, los encargados del \textit{backend} podrán ayudar en la implementación de los \textit{frontend}, encargándose además de modificar y corregir el backend cuando sea necesario. \\

% Lecciones aprendidas
Una de las principales lecciones que hemos aprendido durante el desarrollo del backend, es que debemos gestionar y modularizar el código, no solo en distintos componentes, si no en diferentes ficheros de código. Al estar acostumbrados a proyectos más pequeños, tendemos a escribir código sin pensar en su organización dentro del componente, lo que genera ficheros muy largos y difíciles de leer.\\


Debido a esto, ha sido necesarios realizar varias refactorizaciones del código, lo cuál nos ha servido para familiarizarnos con las diferentes herramientas de refactorización del \textit{IDE} utilizado, \textit{GoLand} en nuestro caso. Hemos mejorado nuestros hábitos de programación, dividiendo el código de un mismo componente en diferentes ficheros más sencillos.\\

% Omitir repeticion del CORS
El tamaño del proyecto ha fomentado también realizar \textit{commits} más concisos y frecuentes.
Además, hemos podido aprender sobre el \textit{CORS} y las peticiones de datos de origen cruzado al servir los clientes web y la API desde dominios diferentes. \\

% Esfuerzos realizados
Para esta entrega Guillermo invirtió un total de 34 horas, encargándose de implementar la fase de refuerzo, las funciones de creación y configuración de partidas, la implementación de la baraja y otras funciones sociales. Por otro lado, Samuel se encargó de desarrollar la fase inicial del juego, la gestión de las listas de amigos, la configuración del \textit{CORS}, el despliegue de la aplicación, la generación automática de documentación con la herramienta \textit{godoc}, el sistema de búsqueda de partidas y las notificaciones. En total, invirtió 35 horas. \\

Como hemos mencionado antes, diferentes tareas fundamentales para el desarrollo del proyecto se realizaron en pareja. Entre ellas podemos destacar la definición del diagrama de estados del juego, la implementación del mapa y el grafo de regiones utilizados , la caché de partidas y el sistema de gestión de sesiones, entre otros. Cabe destacar también que cada uno de los integrantes nos encargamos de realizar programas de prueba para cada uno de los módulos que implementamos.
%\section{Conclusiones}


\clearpage
%\section{Anexos}
%\subsection{Glosario}
%\subsection{Actas de todas las reuniones realizadas}

\newpage
\section{Referencias}
\subsection{Bibliografía}
\printbibliography

\end{document}
